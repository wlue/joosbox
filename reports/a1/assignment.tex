\documentclass[letterpaper]{article}
\usepackage{inconsolata}
\usepackage[T1]{fontenc}
\usepackage[margin=1in]{geometry}
\usepackage[utf8]{inputenc}
\usepackage{hyperref}
\usepackage{soul}
\usepackage{fancyhdr, lastpage}
\usepackage{graphicx}
\pagestyle{fancy}
\fancyhf{}
\usepackage{minted}
\usemintedstyle{pastie}

\newcommand{\spacer}{\vspace{5mm}\hrule\vspace{5mm}}

\makeatletter
\renewcommand{\@maketitle}{
  \begin{center}%
    {\LARGE \@title \par}%
  \end{center}%
  \vspace*{40pt}
  \noindent \Large \@date \par %
  \vspace*{20pt}
  \noindent \Large \@author \par %
  \vfill
  \par
}
\makeatother

% New commands
\newcommand{\assignmentnumber}{1}
\newcommand{\course}{CS 444: Compiler Construction}
\newcommand{\term}{Winter 2013}
\newcommand{\project}{Lab \assignmentnumber: Scanning, Parsing, Weeding, AST Building}
\newcommand{\name}{wlue, cktaylor, psobot}
\newcommand{\wenhao}{wlue(20349659) - Lue, Wen-Hao (wlue@uwaterloo.ca)}
\newcommand{\chris}{cktaylor(20338058) - Taylor, Chris (cktaylor@uwaterloo.ca)}
\newcommand{\peter}{psobot(20334978) - Sobot, Peter (psobot@uwaterloo.ca)}

% Values for template
\title{\course \\ \term \\ \project}
\date{\ul{\textbf{Date of Submission}}: \today}
\author{\ul{\textbf{Submitted by}}: \\ \indent \wenhao \\ \indent \chris \\ \indent \peter}

\rhead{\name{}}
\lhead{\course{} Lab \assignmentnumber}
\cfoot{Page \thepage{} of \protect\pageref{LastPage}}

% Content
\begin{document}

  \maketitle
  \thispagestyle{empty}
  \clearpage

  \tableofcontents
  \thispagestyle{empty}
  \clearpage

  \setcounter{page}{1}

  \clearpage
  \section{Introduction}

  Our CS444 compiler, named {\em Joosbox}, currently scans, parses and
  validates files, outputing a return code of {tt 0} if the file is a valid
  Joos 1W source file, and outputing a return code of 42 otherwise. Upon
  parsing an invalid Joos 1W program, Joosbox will output diagnostic
  information (including line number and character index of invalid tokens) to
  standard error. Joosbox is implemented in Scala, and has a wonderful ASCII-art logo:

  \begin{verbatim}
                                                         ______
                                                        / _____|
                                                       / /
                                                      | |
                                                      | |
                                                      | |
                              +-------------------------------+
                              |                               |
                              |      __   __    __   ____     |
                              |    _(  ) /  \  /  \ / ___)    |
                              |   / \) \(  O )(  O )\___ \    |
                              |   \____/ \__/  \__/ (____/    |
                              |                               |
                              |                               |
                              |            JOOSBOX            |
                              |                               |
                              |    Joos-1W to i386 compiler   |
                              |          100% SCALA           |
                              |                               |
                              |                               |
                              |                               |
                              |                               |
                              |     NOT FROM CONCENTRATE      |
                              |    PRODUCT OF WATERLOO, ON    |
                              |                               |
                              +-------------------------------+
  \end{verbatim}

  \section{Design}

  Joosbox is split into three primary components: the lexer, the parser, and
  the abstract syntax tree builder.

  \subsection{Lexer}

  Joosbox's lexer is based on the Joos1W grammar, a carefully crafted subset
  of the Java 1.3 grammar. Using the feature chart of Joos1W we were able to
  construct a series regexes that would match a given valid token type. From
  each regex, we then constructed a corresponding NFA by hand. To accomodate
  this, we designed our own NFA data structure in Scala and declared the NFAs
  inline in our code.

  \subsection{Parser}

  Joosbox uses an LALR(1) grammar to parse Joos 1W. This grammar is based on a
  LALR(1) grammar for Java 1 itself, with numerous tokens removed and altered to
  match the language features we were supporting as specified by Joos 1W. The
  LALR(1) grammar for Java 1 that we derived our grammer from was found in
  Chapter 19 of {\em The Java Language Specification}. From this collection of
  production rules for a Java 1 LALR(1) grammar, we were able to construct our
  grammar in the form of {\tt joos1w.cfg}.

  The provided {\tt jlalr1.java} program for constructing a parse table from a
  {\tt.cfg} file. With this program we were able to contstruct a LALR(1) parse
  table for our grammar in the form of {\tt joos1w.lr1}. This file was used at
  runtime by Joosbox to parse the Joos1W language.

  To parse the sequence of tokens output by the lexer, Joosbox first discards
  tokens that do not have corresponding semantic value in the grammar (i.e.:
  whitespace, comments, etc) in a ``pre-parse weeding'' step. At this step,
  any tokens that are lexically valid but not semantically valid in Joos
  (including Java constructs that are reserved keywords, but not supported,
  like {\tt throw}) will cause a {\tt SyntaxError} to be raised.

  Once the input tokens are pre-weeded, the LALR(1) algorithm with tree building
  (from CS 241) is used to construct a basic parse tree. The resulting parse 
  tree output is built out of {\tt ParseTreeNode}s, arbitrary expressions
  that can contain zero or more child nodes and an optional value.

  {\tt ParseTreeNode} is implemented as an abstract class in Scala, with all
  of the different types of nodes (corresponding to symbols in the grammar)
  being concrete subclasses. These subclasses are automatically generated
  by a Python script that reads in the {\tt joos1w.lr1} file and outputs
  Scala classes that correspond to each kind of symbol in the grammar. This
  allows Joosbox to make use of Scala's strong built-in type system, adding
  an additional layer of safety to our code.

  \subsection{Abstract Syntax Tree}

  Joosbox accomplishes weeding, or post-parsing validation of semantic input,
  as a validation step during the construction of the Abstract Syntax Tree.

  A single recursive method, {\tt AbstractSyntaxNode.fromParseNode}, is
  responsible for taking a {\tt ParseNode} instance generated by the parser
  and converting it to a {\tt AbstractSyntaxNode}. {\tt AbstractSyntaxNode}s
  are desirable over {\tt ParseNode}s due to their semantic meaning --- while
  {\tt ParseNode}s represent a program, they are fairly generic and do not
  encode the semantic meaning of each node of the tree. For instance, a method
  declaration may be represented by a {\tt ParseNode} with an arbitrary number
  of children, but is more usefully represented as an {\tt AbstractSyntaxNode}
  with dedicated class members for method name, modifiers (public, static,
  etc), member type, formal parameters, and optional body. Having these
  parameters extracted from the node's syntax tree allow us to more easily
  implement methods that require information about the semantic meaning of the
  language constructs.

  To implement weeding during the construction of the Abstract Syntax Tree,
  the {\tt fromParseNode} method includes validations when it creates an {\tt
  AbstractSyntaxNode} from a {\tt ParseNode}. Each different type of {\tt
  AbstractSyntaxNode}  has its own validation checks --- for instance, an {\tt
  InterfaceMemberDeclaration} cannot be static  or final. When this node is
  created, its factory method ensures that this validation passes. If the
  validation  fails, a {\tt SyntaxError} is thrown indicating the type and
  source of the error.

  \section{Challenges}

  %TODO: Talk about bundling together the Scala code into one single jar

  An additional challenge when writing both the lexer and the parser for
  Joosbox was finding a way to use Scala's type system to guarantee type
  safety for our program. Each type of token in the lexer and each type of
  symbol in the parser was implemented as a separate case class in Scala,
  enabling us to perform pattern matching and extraction on tokens and parse
  tree nodes. To enable these features without writing hundreds of repetitive
  classes by hand, two small Scala code generators were written in Python.
  {\tt lexer-type-generator.py} and {\tt parser-type-generator.py} both
  consume input files describing the types of the lexer tokens and parser
  tokens, and use these files to generate valid Scala code that implements
  this type hierarchy. These generators were tricky to write, as most code
  generators are, and required a lot of trial and error to result in correct
  Scala code output. Additionally, the huge number of Scala classes defined by
  the resulting generated code has resulted in longer compile times and larger
  binaries --- but enabled cleaner and safer code to be written.

  %TODO: Talk about code generation from Python to Scala

  \section{Testing}

  For testing, we used $specs2$, a Scala specifications and testing library. We
  ran our tests using $sbt$.

  Joosbox has unit tests for small units of functionality. We
  individually tested each piece of functionality, and small portions of the
  lexer, parser, and AST generator. Our project is separated into separate
  subprojects that deal exclusively with lexing, parsing, and code generation,
  and for each subproject, we have a separate set of specific tests.

  We used TDD (test-driven development) to develop our compiler. For each
  portion of work we started, we wrote passing and failing tests to verify the
  validity of the portion of the program we were working on.

  We imported the Marmoset tests into our local test suite, parsed each test to
  determine if they were passing or failing tests, and generated a unit test for
  each marmoset test. This substantially increased our development speed, as
  this integrated our regular unit testing strategy with the Marmoset tests, and
  also avoided needing to upload a zip file to Marmoset after every code
  iteration.

\end{document}
