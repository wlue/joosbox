\documentclass[letterpaper]{article}
\usepackage{inconsolata}
\usepackage[T1]{fontenc}
\usepackage[margin=1in]{geometry}
\usepackage[utf8]{inputenc}
\usepackage{hyperref}
\usepackage{soul}
\usepackage{fancyhdr, lastpage}
\usepackage{graphicx}
\pagestyle{fancy}
\fancyhf{}
\usepackage{minted}
\usemintedstyle{pastie}

\newcommand{\spacer}{\vspace{5mm}\hrule\vspace{5mm}}

\makeatletter
\renewcommand{\@maketitle}{
  \begin{center}%
    {\LARGE \@title \par}%
  \end{center}%
  \vspace*{40pt}
  \noindent \Large \@date \par %
  \vspace*{20pt}
  \noindent \Large \@author \par %
  \vfill
  \par
}
\makeatother

% New commands
\newcommand{\assignmentnumber}{2-4}
\newcommand{\course}{CS 444: Compiler Construction}
\newcommand{\term}{Winter 2014}
\newcommand{\project}{Labs \assignmentnumber: Name Resolution, Type Checking, Static Analysis}
\newcommand{\name}{wlue, cktaylor, psobot}
\newcommand{\wenhao}{wlue(20349659) - Lue, Wen-Hao (wlue@uwaterloo.ca)}
\newcommand{\chris}{cktaylor(20338058) - Taylor, Chris (cktaylor@uwaterloo.ca)}
\newcommand{\peter}{psobot(20334978) - Sobot, Peter (psobot@uwaterloo.ca)}

% Values for template
\title{\course \\ \term \\ \project}
\date{\ul{\textbf{Date of Submission}}: \today}
\author{\ul{\textbf{Submitted by}}: \\ \indent \wenhao \\ \indent \chris \\ \indent \peter}

\rhead{\name{}}
\lhead{\course{} Assignment \assignmentnumber}
\cfoot{Page \thepage{} of \protect\pageref{LastPage}}

% Content
\begin{document}

  \maketitle
  \thispagestyle{empty}
  \clearpage

  \setcounter{page}{1}

  \clearpage
  \section{Introduction}

  Our Joos1W compiler, named {\em Joosbox}, currently performs the following
  operations on all programs passed in on the command line:

  \begin{itemize}
    \item Scanning
    \item Parsing
    \item Weeding
    \item Name Resolution
    \item Hierarchy Checking
    \item Type Checking
    \item Static Analysis
  \end{itemize}

  When passed a valid Joos1W program, Joosbox will print nothing on standard
  error or standard output, and will return a {\tt 0} return code. Upon
  parsing an invalid Joos1W program, Joosbox will output diagnostic
  information (including line number and character index of invalid tokens, if
  available) to the standard error stream. Joosbox will then return {\tt 42}.

  Joosbox is implemented in Scala, and makes use of only four libraries ---
  the Scala standard library, the {\tt SBT} built tool, the {\tt Specs2}
  testing framework, and Apache Commons Lang. This document outlines the
  design of Joosbox, enumerates the significant challenges encountered during
  its construction, and describes our group's testing process.

  \section{Design}

  \section{Challenges}

  \section{Testing}

\end{document}
